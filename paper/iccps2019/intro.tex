\section{Introduction}
\label{sec:intro}
A cyber-physical system (CPS) consists of computing devices communicating with one another and interacting with the physical world via sensors and actuators. Increasingly, such systems are everywhere, from smart buildings to autonomous vehicles to mission-critical military systems. 
%
The rapidly expanding field of CPSs precipitated a corresponding growth in security concerns for these systems. The increasing amount of software, communication channels, sensors and actuators embedded in modern CPSs make them likely to be more vulnerable to both cyber-based and physical-based attacks~\cite{wan2015security, wasicek2014aspect, kocher2004security, al2015design, gamage2010enforcing}. As an example, \emph{sensor spoofing} attacks to CPSs become prominent, where a hacker can arbitrarily manipulate the sensor measurements to compromise secure information or to drive the system toward unsafe behaviors. Such attacks have successfully disputed the braking function of the anti-lock braking systems~\cite{Shoukry2013,al2015design}, and compromised the insulin delivery service of a diabetes therapy system~\cite{li2011hijacking}. Alternatively, attackers can gain access to communication channels to either manipulate the switching behavior of a smart power grid~\cite{liu2011class} or disable the brake system of a modern vehicle~\cite{koscher2010experimental}. 
%
Therefore, it is essential for a designer to build a CPS model with a resilient capability to cope with different attack scenarios. However, many CPSs are developed today without considering resiliency at all. Model-based system developers often neglect or incompletely consider incorporating resilient patterns in their designs. 
%
%
Generally, constructing a behavioral model at design time that offers resiliency for all kinds of attacks and failures is notoriously difficult as it requires investigating many sub-components in an integrated manner, especially for a large-scale CPS.
%
%A general solution would be to formally specify safety properties of a CPS that protect the system against possible adversarial attacks using formalisms such as signal temporal logic (STL)~\cite{maler2004monitoring} and to then iteratively improve the design using \emph{falsification}~\cite{nghiem2010monte}, which would automatically identify vulnerabilities in the design. 

%Model-based design offers a promising approach for assisting developers to build a reliable and secured CPSs in a systematic manner~\cite{alur2015principles,lee2000s,barthe2004secure}. 
%
Traditionally a model of a CPS consists of block diagrams describing the system architecture and a combination of state machines and differential equations describing the system dynamics~\cite{alur1995algorithmic}. Suppose the designer has initially constructed a model of a CPS that satisfies correctness requirements, but at a later stage, this correctness guarantee is invalidated, possibly due to the emergence of new requirements, or adversarial attacks on sensors, or violation of environment assumptions. There is currently lack of an automated mechanism that can efficiently repair the initial design and provide resilience.  
%
%
Many significant research have been introduced to build resilient CPSs such as the approach proposed in~\cite{fitzgerald2012rigorous} that can be used to design a resilient CPS through co-simulation of discrete-event models, a modeling and simulation integration platform for secure and resilient CPS based on attacker-defender games proposed in~\cite{koutsoukos2018sure} with the corresponding testbed introduced in~\cite{neema2018integrated}, and the resilience profiling of CPSs presented in~\cite{jackson28resilience}. Although these approaches can leverage the modeling and testing for a resilient CPS, they do not offer a model repair mechanism as well as a generic approach to design a resilient pattern when vulnerabilities are discovered. In most of the case, a designer needs to rebuild the system from scratch, which requires a lot of time and efforts. 
%
%
 %Moreover, there is a lack of a generic method for a designer to specify a resilient pattern.  
%
 %
%
%

%Suppose the designer has initially constructed a model of a CPS that satisfies correctness requirements, but at a later stage, this correctness guarantee is invalidated, possibly due to the emergence of new requirements, or adversarial attacks on sensors, or violation of environment assumptions. 
%
%

In this paper, we propose a new methodology and an associated toolkit, which we call \toolreaffirm, to assist a designer in repairing the partial behavior model to generate the completed behavior model with resilience. 
%
The proposed technique is relied on designing a collection of \emph{potential edits} (or \emph{resilient patterns}) to the original model to solve the \emph{model synthesis problem}. 
%
For example, a conservative way ensuring safety upon encountering an unexpected or hazardous situation is to hand over control to a baseline safety controller.
%
\toolreaffirm takes the original design and known resilient patterns as inputs and then synthesizes the completed model that satisfies the system requirements.  
%
The resilient version generated by \toolreaffirm may have additional modes of operation as well as new transitions between different modes, and as a result, has resilient behaviors derived from a counterexample returned by the \emph{falsification} tool embedded within \toolreaffirm. 
%
Since a counterexample characterizes undesirable, or \emph{shall not}, situations, it can naturally describe how a system should respond when a particular sensor fails, or a previously unanticipated attack is identified. The model synthesizer within \toolreaffirm can automatically integrate such a counterexample with state-machine based models thus allowing an incremental design to support resiliency.
%
%
We also provide a new \emph{model transformation language} for hybrid systems, called HATL. Such a transformation language allows a designer to specify a resilient pattern in a generic manner, without knowing the internal structures of a system. To the best of our knowledge, this transformation language is the first effort to design a programmable pattern for the modeling and repairing of resilient CPSs. 

%For example, a conservative way ensuring safety upon encountering an unexpected or hazardous situation is to hand over control to a baseline safety controller.

%Design and implementation of these tools will require theoretical advances in terms of rigorous formalization, computational engines, and heuristics for scalability.
%the goal of \toolreaffirm is to facilitate the integration of evolving resiliency requirements in model-based design and analysis for CPSs.  
Overall, the main contributions of the paper are summarized as follows.
%
\begin{enumerate}[leftmargin= 2 em]
\item the methodology to facilitate the integration of evolving resiliency requirements in model-based design and analysis for CPSs,
\item the end-to-end design and implementation of the tool-chain, which integrates the model transformation and the model synthesis tools to automatically repair CPS models,
\item the design and implementation of the model transformation language for specifying resilient patterns used to repair CPS models,
\item the applicability of our approach on two proof-of-concept case studies where the resilient CPS models can be automatically constructed to mitigate practical attacks.
\end{enumerate}
%
We anticipate that our methodology proposed along with \toolreaffirm for automatically conducting the model repair and specifying the patterns of resiliency will be contributions of significant interest to the research community in the design of resilient CPSs.
%
%
The remainder of the paper is organized as follows.~\secref{overview} presents an overview of our proposed methodology through the running example of an adaptive cruise control system (ACC), and introduces the architecture of \toolreaffirm.~\secref{transformation} describes our model transformation language used to design a resilient pattern for hybrid systems.~\secref{synthesis} presents the model synthesizer of \toolreaffirm.~\secref{result} presents two proof-of-concept case studies that illustrate the capability of \toolreaffirm in automatically repairing the original models of a) the ACC system under sensor spoofing attacks and b) the smart power grid system under sliding-mode attacks.~\secref{rw} reviews the related works to \toolreaffirm.~\secref{conclude} concludes the paper and presents future research directions for the proposed work.






