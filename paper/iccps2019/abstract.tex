\begin{abstract}
\label{sec:abstract}
%%
Model-based design offers a promising approach for assisting developers to build reliable and secure cyber-physical systems (CPSs) in a systematic manner. In this methodology, a designer first constructs a model, with mathematically precise semantics, of the system under design, and performs extensive analysis with respect to correctness requirements before generating the implementation from the model.
%
However, as new vulnerabilities are discovered, requirements evolve aimed at ensuring resiliency. 
%
There is currently a shortage of an inexpensive, automated mechanism that can effectively repair the initial design, and a model-based system developer regularly needs to redesign and reimplement the system from scratch.
%
%Current methodology demands an expensive, and at times infeasible, redesign and reimplementation of the system from scratch.
\JW{Fixed:This is a very strong statement -- maybe weaken it a bit to avoid angering reviewers.}
%
In this paper, we propose a new methodology along with a Matlab toolkit called \toolreaffirm to facilitate the integration of evolving cyber-physical resiliency  requirements \JW{Fixed: resiliency requirements is very vague -- can we be more specific?  Maybe cyber-physical resiliency requirements?} in model-based design for CPSs.
%
\toolreaffirm contains two main modules, a model transformation, and a model synthesizer. 
%
The first module applies a given resiliency pattern specified as a model transformation script to an original model and produces a candidate resilient model that contains parameters for which the values need to be synthesized. 
%
The second module built on top of the falsification tool Breach takes a parametrized model as an input and performs a search over the space of parameter domains to find the best values for which the input safety requirement, expressed as a Signal Temporal Logic formula, satisfied.
%
%synthesis of parameter values that makes the requirement, expressed as a formula in the Signal Temporal Logic, satisfied. %It internally uses the falsification tool Breach properly to search over the space of parameter values and find the best values for which the input safety property is violated. %Based on the counterexample produced by Breach, parameter ranges are tightened, and the process repeats until the property is satisfied. 
\JW{Fixed: The discussion of the second module seems a bit long -- we have the whole paper to describe the specifics of the tool.}
%
%
To evaluate the proposed approach, we use \toolreaffirm to automatically synthesize resilient models for an adaptive cruise control (ACC) system under GPS sensor spoofing attacks, and for a single-machine infinite-bus (SMIB) system under a sliding-mode switching attack.
%
\end{abstract} 	