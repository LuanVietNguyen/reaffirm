\section{Adaptive Cruise Control System}
\seclabel{example}


\begin{figure}[t!]%
	\centering%
	\begin{adjustbox}{max size={0.99\columnwidth}{0.75\textheight}}%
	\begin{tikzpicture}[>=stealth',shorten >=1pt,auto,node distance=8cm,font=\Large]
		%\tikzstyle{every state}=[minimum size=3cm,font=\normalsize]
		\tikzstyle{every state}=[font=\Large,rectangle,rounded corners, minimum size=5cm]
		\node[state] (speed)      			{\makecell[c]{$\textbf{Speed Control}$\\\\
		$\begin{aligned}
		\dot{d} & = \vlead - v \nonumber \\
		\dot{v} & = 2\vlead - v - \hat{v} \nonumber \\
		\dot{\hat{d}} & = \vlead - \hat{v} + 10(\drad-\hat{d})\nonumber \\
		\dot{\hat{v}} & = 2\vlead - 3\hat{v} + \frac{1}{2} (\ngps + \nenc) \nonumber\\
		\end{aligned}$\\\\
		$\hat{d} \geq 10 + 2\hat{v}$
		}};
		\node[state] (space) [right of=speed]	{\makecell[c]{$\textbf{Spacing Control}$\\\\
		$\begin{aligned}
		\dot{d} & = \vlead - v \nonumber \\
		\dot{v} & = 2\vlead - v - \hat{v} -\frac{1}{4}(10 + 2\hat{v}-\hat{d}) \nonumber \\
		\dot{\hat{d}} & = \vlead - \hat{v} + 10 (\drad-\hat{d})\nonumber \\
		\dot{\hat{v}} & = 2\vlead - 3\hat{v} + \frac{1}{2} (\ngps + \nenc) -\frac{1}{4}(10 + 2\hat{v}-\hat{d}) \nonumber\\
		\end{aligned}$\\\\
		$\hat{d} < 10 + 2\hat{v}$
		}};
		
		\node (speedtospace) [above=6mm of speed,xshift=40mm]{\makecell[c]{$\hat{d} < 10 + 2\hat{v}$}};
		%$\dot{d} = \vlead - v$\\$\dot{v} = 2\vlead - v - \hat{v} -\frac{1}{4}(10 + 2\hat{v}-\hat{d}$\\$\dot{\hat{d}} = \vlead - \hat{v} + 10(\drad-\hat{d})$\\$\dot{\hat{v}} = 2\vlead - 3\hat{v} + \frac{1}{2} (\ngps + \nenc) -\frac{1}{4}(10 + 2\hat{v}-\hat{d}$}}; 
		
		\node[coordinate] (c1) [above=5mm of speed.north] {};
		\node[coordinate] (c2) [above=5mm of space.north] {};
		\draw[->]			(speed.north) -- (c1) -- (c2) to (space.north);
		
		\node (spacetospeed) [below=6mm of space,xshift=-40mm]{\makecell[c]{$\hat{d} \geq 10 + 2\hat{v}$}};

		\node[coordinate] (c3) [below=5mm of space.south] {};
		\node[coordinate] (c4) [below=5mm of speed.south] {};
		\draw[->]			(space.south) -- (c3) -- (c4) to (speed.south);
		%
		%\path[->]			(speed.north) edge[bend left] node{\makecell[c]{$\mathit{OD} \leq \mathit{OD}_t \wedge t_m \geq t_\text{mOn} $\\$t_m' := 0$}} (space.north); 
		%\path[->]			(space.south)	edge[bend left] node{\makecell[c]{$\mathit{OD} \geq \mathit{OD}_t \wedge t_m \geq t_\text{mOff} $\\$t_m' := 0$}} (speed.south); 
	\end{tikzpicture}%
	\end{adjustbox}%
	\caption{Original ACC model}%
	\figlabel{original}%
\end{figure}

To illustrate and motivate our proposed approach, this subsection presents a case study of a control system that aims to safely adapt a vehicle speed, namely, an adaptive cruise control (ACC) system. 
%
%In the following, we first present the original (legacy) system which satisfies an initial functional safety requirement. 
%
Assume that a designer has previously modeled an ACC system as a combination of the vehicle dynamics and an control module. This system has been designed to satisfy safety requirements pertaining to vehicle spacing that varies with vehicle speed. In the initial design, resiliency was not considered. In the following, we first describe the ACC system as originally designed, which will (in the subsequent discussion) be adapted to satisfy a resiliency requirement.

\subsection{Original ACC Model}
The ACC system operates in two modes: \emph{speed control} and \emph{spacing control}. In speed control, the host car travels at a driver-set speed. In spacing control, the host car aims to maintain a safe distance from the lead car. The ACC system decides which mode to use based on the real-time sensor measurements. For example, if the lead car is too close, the ACC system switches from speed control to spacing control. 
%
Similarly, if the lead car is further away, the ACC system switches from spacing control to speed control. In other words, the ACC system makes the host car travel at a driver-set speed as long as a safe distance is maintained.

The vehicle has two states in which $d$ is the distance to the lead car, and $v$ is the speed of the host vehicle. These states evolve according to $\dot{d} = \vlead - v$ and $\dot{v} = -v + u$, 
%
where $\vlead$ is the speed of the lead vehicle  and $u$ is the control input.  In this example, we assume that the lead vehicle speed is known exactly (\eg it is communicated between vehicles). Conceptually, the dynamics for $d$ represent that the derivative of the relative distance is the difference between the lead vehicle speed and the host vehicle speed.  The dynamics for the host vehicle speed indicate that as the vehicle speed increases, it takes more acceleration (\ie force) provided by the controller (\ie engine) to maintain speed. 

An ACC equipped vehicle has sensors that measure its velocity $v$ via noisy wheel encoders, $\venc = v + \nenc$, and a noisy GPS sensor, $\vgps = v+\ngps$, where $\nenc$ and $\ngps$ denote the encoder and GPS noisy, respectively. 
%
%
Additionally, the ACC system has a radar sensor that measures the distance to a (potential) lead vehicle, $\drad = d+\nrad$. 
%
To design a control law, we need to estimate the state of the vehicle (\ie we need an estimate of $d$ and $v$, which we will call $\hat{d}$ and $\hat{v}$). To estimate the distance and velocity, we employ state estimators
%
\begin{align}
\dot{d} & = \vlead - \hat{v} + 10(\drad - \hat{d}) \nonumber \\
\dot{v} & = -\hat{v} + u +  \frac{1}{2} ((\vgps + \venc) - \hat{v}) \nonumber
\end{align}
%
%
To implement a controller, a control law based on the state estimates in speed and distance control modes are given: 

\begin{align}
u_s & = \vlead - (\hat{v}-\vlead) \nonumber \\
u_d & = \vlead - (\hat{v}-\vlead) - \frac{1}{4} (d_{ref} - \hat{d}) \nonumber
\end{align}
%
%
where $u_s$ is the control law in the speed mode, $u_d$ is the control law in the spacing mode, and $d_{ref} = 10 + 2\hat{v}$. These control laws incorporate a reference velocity $\vlead$, which can be thought of as constant gain that depends on the lead (or desired) vehicle velocity, while the other terms depend on the deviation of the lead vehicle and and ego vehicle states. Switching between modes is handled by monitoring the state estimates. The designer models the complete ACC system (including vehicle dynamics) as a hybrid system, illustrated in \figref{original}. Here, the transition from speed control to spacing control occurs when the estimate of the distance is less than twice the estimated safe distance, \ie $\hat{d} < 10 + 2\hat{v}$. A similar condition is provided for transitioning from spacing control to speed control, \ie $\hat{d} \geq 10 + 2\hat{v}$. 

In this example, the functional safety specification of the system is specified that $d$ should always be greater than $d_{safe} = 5 + v$. Assume that the designer has verified the ACC system safety requirement under the scenario when $d(0) \geq 20$, $v(0) \leq 30$, $|d(0) - \hat{d}(0)| \leq 1$, $|v(0) - \hat{v}(0)| \leq 1$ , $\vlead \geq 0$, $|\nenc| \leq 0.05$ and $|\ngps| \leq 0.05$.
%
 %Next we add a new resilience requirement that is determined to be violated by the original system. Finally, we adapt the original system such that the resilience requirement is satisfied.
After designing the initial ACC system, it is determined that the GPS sensor can be \emph{spoofed} \cite{tippenhauer2011requirements, kerns2014unmanned}. GPS spoofing occurs when incorrect GPS packets (possibly sent by a malicious attacker) are received by the GPS receiver. In the ACC system, this allows an attacker to arbitrarily change the GPS velocity measurement. 
%
Thus, a new scenario occurs when the assumption that $\ngps \leq 0.05$ is omitted, and the new assumption is $|\ngps| \leq \infty$.
As a result, the safety specification could be violated under the GPS sensor attacks, and a designer needs to repair the original model with a resilient pattern. 

\subsection{Updated Resilient ACC Model}
%
To provide resilience against the GPS attacks, a potential strategy is to ignore the GPS value in resilient modes and use only the wheel encoders to estimate velocity. Since the ACC system has redundancy in the sensory information of its estimated velocity, the model synthesizer can repair the model by replacing the GPS velocity measurement with the wheel encoder velocity measurement when the GPS measurement significantly deviates from the wheel encoder measurement.
%
%
Thus, the proposed fix then is first to create a resilient copy of the original model where the controller simply ignores the GPS reading as it can no longer be trusted. Then, adding new transitions from the legacy speed and spacing modes of the original model to the new resilient speed and spacing modes of the copy that uses only the wheel encoder as a velocity measurement source.
%

We note that this transformation is generic, that is, it can be applied in a uniform manner to any given model simply by creating a duplicate version of each original mode and transition, copying the dynamics in each mode, but without a reference to the variable $\ngps$. 
%
%
Observe that while it would be possible to use only the wheel encoder all the time, a better velocity estimate must be obtained by using an average velocity measurement (from both the GPS and wheel encoders) when the GPS sensor is performing within nominal specifications. The main analysis question is when should the model switch from original copy to this resilient copy.  In this example, we model this switching condition as $|\ngps -\nenc| \geq \theta$, where $\theta$ is an unknown parameter. The \emph{parameterized} model is shown in \figref{updated}, where the variable $\ngps$ is replaced by the variable $\nenc$ in the resilient speed and spacing control modes.


\begin{figure}[t!]%
	\centering%
	\begin{adjustbox}{max size={0.99\columnwidth}{0.75\textheight}}%
	\begin{tikzpicture}[>=stealth',shorten >=1pt,auto,node distance=8cm,font=\Large]
		%\tikzstyle{every state}=[minimum size=3cm,font=\normalsize]
		\tikzstyle{every state}=[font=\Large,rectangle,rounded corners, minimum size=5cm]
		\node[state] (speed)      			{\makecell[c]{$\textbf{Speed Control}$\\\\
		$\begin{aligned}
		\dot{d} & = \vlead - v \nonumber \\
		\dot{v} & = 2\vlead - v - \hat{v} \nonumber \\
		\dot{\hat{d}} & = \vlead - \hat{v} + 10(\drad-\hat{d})\nonumber \\
		\dot{\hat{v}} & = 2\vlead - 3\hat{v} + \frac{1}{2} (\ngps + \nenc) \nonumber\\
		\end{aligned}$\\\\
		$\hat{d} \geq 10 + 2\hat{v}$
		}};
		\node[state] (space) [right of=speed]	{\makecell[c]{$\textbf{Spacing Control}$\\\\
		$\begin{aligned}
		\dot{d} & = \vlead - v \nonumber \\
		\dot{v} & = 2\vlead - v - \hat{v} -\frac{1}{4}(10 + 2\hat{v}-\hat{d}) \nonumber \\
		\dot{\hat{d}} & = \vlead - \hat{v} + 10 (\drad-\hat{d})\nonumber \\
		\dot{\hat{v}} & = 2\vlead - 3\hat{v} + \frac{1}{2} (\ngps + \nenc) -\frac{1}{4}(10 + 2\hat{v}-\hat{d}) \nonumber\\
		\end{aligned}$\\\\
		$\hat{d} < 10 + 2\hat{v}$
		}};
		
		\node[state] (res_speed) [below of=speed]  {\makecell[c]{$\textbf{Resilient Speed Control}$\\\\
		$\begin{aligned}
		\dot{d} & = \vlead - v \nonumber \\
		\dot{v} & = 2\vlead - v - \hat{v} \nonumber \\
		\dot{\hat{d}} & = \vlead - \hat{v} + 10(\drad-\hat{d})\nonumber \\
		\dot{\hat{v}} & = 2\vlead - 3\hat{v} + \frac{1}{2} (\nenc + \nenc) \nonumber\\
		\end{aligned}$\\\\
		$\hat{d} \geq 10 + 2\hat{v}$
		}};
		
		\node[state] (res_space) [right of=res_speed]	{\makecell[c]{$\textbf{Resilient Spacing Control}$\\\\
		$\begin{aligned}
		\dot{d} & = \vlead - v \nonumber \\
		\dot{v} & = 2\vlead - v - \hat{v} -\frac{1}{4}(10 + 2\hat{v}-\hat{d}) \nonumber \\
		\dot{\hat{d}} & = \vlead - \hat{v} + 10 (\drad-\hat{d})\nonumber \\
		\dot{\hat{v}} & = 2\vlead - 3\hat{v} + \frac{1}{2} (\nenc + \nenc) -\frac{1}{4}(10 + 2\hat{v}-\hat{d}) \nonumber\\
		\end{aligned}$\\\\
		$\hat{d} < 10 + 2\hat{v}$
		}};
		
		
		\node (speedtospace) [above=6mm of speed,xshift=40mm]{\makecell[c]{$\hat{d} < 10 + 2\hat{v}$}};
		%$\dot{d} = \vlead - v$\\$\dot{v} = 2\vlead - v - \hat{v} -\frac{1}{4}(10 + 2\hat{v}-\hat{d}$\\$\dot{\hat{d}} = \vlead - \hat{v} + 10(\drad-\hat{d})$\\$\dot{\hat{v}} = 2\vlead - 3\hat{v} + \frac{1}{2} (\ngps + \nenc) -\frac{1}{4}(10 + 2\hat{v}-\hat{d}$}}; 
		
		\node[coordinate] (c1) [above=5mm of speed.north] {};
		\node[coordinate] (c2) [above=5mm of space.north] {};
		\draw[->]			(speed.north) -- (c1) -- (c2) to (space.north);
		
		\node (spacetospeed) [below=6mm of space,xshift=-40mm]{\makecell[c]{$\hat{d} \geq 10 + 2\hat{v}$}};

		\node[coordinate] (c3) [below=5mm of space.south] {};
		\node[coordinate] (c4) [below=5mm of speed.south] {};
		\draw[->]			(space.south) -- (c3) -- (c4) to (speed.south);
		
		\node (res_speedtospace) [above=6mm of res_speed,xshift=40mm]{\makecell[c]{$\hat{d} < 10 + 2\hat{v}$}};
		%$\dot{d} = \vlead - v$\\$\dot{v} = 2\vlead - v - \hat{v} -\frac{1}{4}(10 + 2\hat{v}-\hat{d}$\\$\dot{\hat{d}} = \vlead - \hat{v} + 10(\drad-\hat{d})$\\$\dot{\hat{v}} = 2\vlead - 3\hat{v} + \frac{1}{2} (\ngps + \nenc) -\frac{1}{4}(10 + 2\hat{v}-\hat{d}$}}; 
		
		\node[coordinate] (c5) [above=5mm of res_speed.north] {};
		\node[coordinate] (c6) [above=5mm of res_space.north] {};
		\draw[->]			(res_speed.north) -- (c5) -- (c6) to (res_space.north);
		
		\node (res_spacetospeed) [below=6mm of res_space,xshift=-40mm]{\makecell[c]{$\hat{d} \geq 10 + 2\hat{v}$}};

		\node[coordinate] (c7) [below=5mm of res_space.south] {};
		\node[coordinate] (c8) [below=5mm of res_speed.south] {};
		\draw[->]			(res_space.south) -- (c7) -- (c8) to (res_speed.south);
		
		\node[coordinate] (c9) 	[left=5mm of speed.south] {};
		\node[coordinate] (c10) [left=5mm of res_speed.north] {};
		\node[coordinate] (c11) [right=5mm of space.south] {};
		\node[coordinate] (c12) [right=5mm of res_space.north] {};
		
		%\draw[->]			(speed.south)[right=1mm to (res_speed.north);
		%\draw[->]			(space.south) to (res_space.north);
		
		\path[->]			(c9) edge node[xshift=-29mm]{\makecell[c]{$|\ngps -\nenc| \geq \theta$}}(c10); 
		\path[->]			(c11) edge node[xshift=1mm]{\makecell[c]{$|\ngps -\nenc| \geq \theta$}} (c12); 
		
		%
		%\path[->]			(speed.north) edge[bend left] node{\makecell[c]{$\mathit{OD} \leq \mathit{OD}_t \wedge t_m \geq t_\text{mOn} $\\$t_m' := 0$}} (space.north); 
		%\path[->]			(space.south)	edge[bend left] node{\makecell[c]{$\mathit{OD} \geq \mathit{OD}_t \wedge t_m \geq t_\text{mOff} $\\$t_m' := 0$}} (speed.south); 
	\end{tikzpicture}%
	\end{adjustbox}%
	\caption{Updated Resilient ACC model}%
	\figlabel{updated}%
\end{figure}