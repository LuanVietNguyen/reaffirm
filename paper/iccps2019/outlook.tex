%!TEX root = main.tex
\section{Conclusion and Future Work}
%\section{Conclusion}
\seclabel{conclude}
%
In this paper, we have presented a new methodology and the toolkit \toolreaffirm that effectively assist a designer to repair CPS models under unanticipated attacks automatically. 
%
The model transformation tool takes a resiliency pattern specified in the transformation language HATL and generates a new model including unknown parameters whose values can be determined by the synthesizer tool such that the safety requirement is satisfied.
%
We demonstrated the applicability of \toolreaffirm by using the toolkit to efficiently repair CPS models under realistic attacks including the ACC models under the GPS sensor spoofing attack and the SMIB models under the sliding-model attack.

\vspace{0.5em}
\noindent
{\bf Future Work.} We first plan to perform a systematic classification of common attacks of various types of CPSs based on the work presented in~\cite{humayed2017cyber} and then develop an extensible library of resiliency patterns that encapsulates general mitigation strategies to repair CPS models under these common attacks.
%
Beside using Breach, we also intend to extend the model synthesizer of \toolreaffirm to incorporate various verification tools such as Flow* and dReach to verify the repaired model formally.
%
From the application perspective of \toolreaffirm, we are interested in investigating the SLSF model of a missile guidance system provided by Matlab\footnote{The model is available at \url{https://www.mathworks.com/help/simulink/examples/designing-a-guidance-system-in-matlab-and-simulink.html}.}, the attack scenarios that destabilizes or drives the system toward unsafe behaviors and the resiliency patterns that can be used to repair the SLSF model for improving resiliency.

%
%\vspace{0.25em}
%\paragraph*{Future Work} %There are several directions for the future work of HyperSTL. 
%We first plan to introduce a library of HyperSTL fomulae that encapsulates different general classes of hyperproperties of CPS including those presented in this paper.
%%
%Second, the falsification algorithm of HyperSTL proposed in the paper is incomplete as it relies on self-composition (i.e. making copies of a system) and only falsifies a restricted class of hyperproperties. 
%%
%Thus, extending the falsification algorithm to bypass self-composition to falsify more interesting hyperproperties is planned. Also, the monitoring algorithms of HyperLTL recently proposed in~\cite{brett2017rewriting, agrawal2016runtime} could be applied to HyperSTL.
%%
%%On the other hand, we will also leverage the testing framework of HyperSTL to falsify a hyperproperty with finite-simulation guarantees.% and evaluate the performance of a bug-finding technique using industrial Simulink models.
%
%
%%In future work, we will improve a monitoring algorithm for HyperSTL and apply it to find falsifying traces for hyperproperties of industrial Simulink models.
