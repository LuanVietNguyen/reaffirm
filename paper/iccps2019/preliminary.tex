\section{Preliminaries}
\seclabel{pre}


%\subsection{Protocol Completion}



\subsection{Parameterized Hybrid Input-Output Automata}
Hybrid automata~\cite{alur1995algorithmic} are a modeling formalism popularly used to model CPSs which include both continuous dynamics and discrete state transitions. A hybrid automaton is essentially a FSM extended with a set of real-valued variables evolving continuously over intervals of real-time~\cite{alur1995algorithmic}. In our approach, we consider a CPS that can be formally represented as parametrized hybrid input-output automata (PHIOA)\footnote{In this paper, we only consider \emph{deterministic} systems, where the system produces the same output for a given input. Note the contrast with {\em stochastic} systems in which one or more elements of the system have randomness associated with them; for example, the value of some system parameter may be extracted from a probability distribution. As a result, the stochastic system may yield different outputs for a given input.}, where the parameters representing arbitrary values which do not change during one system execution~\cite{frehse2008counterexample, schwarz2011modelling}. 
%\noindent
%{\bf Hybrid Automata.}
\begin{definition}[Parameterized Hybrid Input-Output Automata]
A $\AutomatonH$ is a tuple, $\AutomatonH$ $\deq$ $\langle$$\Varset$, $\Locset$,  $\Transset$, $\Trajectoryset$, $\Initset$$\rangle$, where
%
%\begin{itemize}[leftmargin= 1 em]
\begin{itemize}[leftmargin= 2em]
%\begin{inparaenum}[(a)]
%
%
%\item $\mathtt{Lab}$: a finite set of synchronization labels,
%
\item $\Varset$: the finite set of $n$ continuous, real-valued variables; we have $\Varset = \X \cup \P$, where $\X \in \Realn$ is the set of $n$ state variables and $\P\in\Realp$ is the set of $p$ parameters. Moreover, $\Varset$ is the disjoint of the set of input variables $\I$ and the set of output variables $\O$, and $\Q \deq \Locset \times \Realn$ is the state space.
%
\item $\Locset$: the finite set of discrete modes.  For each mode $\mode \in \Locset$, $\mode.\inveg \subseteq \Real^{n+p}$ denotes the invariant of mode $\mode$, and $\mode.\floweg \subseteq \Realn$ describes the continuous dynamics.
%and $\forall x \in \Varset$, a \emph{valuation} $\val{x} \in \Real$ is a function mapping $x$ to a point in its type---here, $\Real$; and $\Q \deq \Locset \times \Realn$ is the state space. 
%\item $\VarInput$: the set of input variables. 
%\item $\VarOutput$: the set of output variables.
%
%\item $\Parameterset$: the set of parameters. 
%
%
%
%\item $\Invset$: the finite set of invariants. For each mode $\mode \in \Locset$, $\mode.\inveg \subseteq \Realn$ denotes the invariant $\inveg \in \Invset$ of mode $\mode$. 
%
%\item $\Flowset$: the finite set of ordinary differential inclusions. For each $\floweg\in \Flowset$, $\mode.\floweg \subseteq \Realn$ describes the continuous dynamics in each mode $\mode \in \Locset$.
%
\item $\Transset$: the finite set of transitions between modes.
%
Each transition is a tuple $\tau \deq \tuple{\mode, \mode', \guard, \reset}$, where $\mode$ is a source mode and $\mode'$ is a target mode that may be taken when a guard condition $\guard$ is satisfied, and the post-state is updated by an update map $\reset$.
%
\item $\Trajectoryset$: a finite set of continuous trajectory models the valuations of state variables over an interval of real time $[0, T]$. Let $\val{x}_t$ be the valuations of variable $x$ at time points $t$, $\forall t \in [0, T]$, $\forall x \in \X$, $\exists m \in \Locset$, a trajectory $\gamma \in \Trajectoryset$ is a mapping function $\gamma: [0, T] \rightarrow \val{\X}$ such that:
\begin{itemize}
\item $\val{x}_t = \val{x}_0 + \int_{\delta = 0}^{t} m.\floweg(x) d\delta$, and %where $\Delta_0$ is a valuation of variable $x$ at $t = 0$
\item $\val{x}_t \models m.\inveg$ for all $t \in [0, T]$.
%
\end{itemize}
%
\item $\Initset$ is an initial condition, and $\Initset \subseteq \Q$.
%\end{inparaenum}
\end{itemize}
\end{definition}
% 
We use the dot (.) notation to refer to different components of tuples \eg $\AutomatonH.\Transset$ refers to the transitions of automaton $\AutomatonH$ and $\tau.\guard$ refers to the guard of a transition $\tau$.
%\end{definition}

The semantics of a PHIOA $\AutomatonH$ are defined in terms of executions, which are sequences of states. Given the set of parameter value $P_0\in \Realp$, an \emph{execution} of $\AutomatonH$ w.r.t $P_0$ is a sequence $\pi_{P_0} \deq \s_0 \rightarrow \s_1 \rightarrow \s_2 \rightarrow \dotline$, where $\s_0 \in \Initset$ is an initial state, and $\s_i \rightarrow \s_{i+1}$ is the update from the current-state $\s_i$ to the post-state $\s_{i+1}$, that is specified by the transition relations of $\AutomatonH$ including:
\begin{inparaenum}[(a)]
\item a discrete transition that describes the instantaneous state update, or
\item a continuous trajectory that represents the state update over a real time interval (see \cite{david2010timed,frehse2008counterexample} for more details).
\end{inparaenum}
%
%
We say a state $\s_k$ is \emph{reachable} from an initial state $\s_0$ w.r.t $P_0$ if there exists an execution $\pi_{P_0} \deq \s_0 \rightarrow \s_1 \rightarrow \dotline \rightarrow \sk$. We denote $\exec(\AutomatonH)$ and $\reach(\AutomatonH)$  as the set of all executions and the set of all reachable states of $\AutomatonH$, respectively.

\begin{definition}[Verification Problem]
Given a safety specification $\varphi$ which is a formula over $\Locset$ and $\Varset$ that describes a set of states $\ds{\varphi} \subseteq \Q$, where $\ds{\cdot}$ is the set of states satisfying $\varphi$, a PHIOA automaton $\AutomatonH$ satisfies the specification $\phi$, denoted $\AutomatonH \models \phi$ if and only if $\reach(\AutomatonH) \subseteq \ds{\varphi}$.
\end{definition}

If a PHIOA automaton $\AutomatonH$ does not satisfy a safety specification $\varphi$, there exists an execution $\pi_{P_0} \in \exec(\AutomatonH)$ w.r.t to some parameter values $P$ in which some state $\si \in \pi_{P_0}$ violates $\varphi$, \ie $\si \notin \ds{\varphi}$. We call such an execution as a \emph{counterexample}. Searching for a counterexample of $\AutomatonH$ with respect to a safety property $\varphi$ is considered as solving the falsification problem of $\AutomatonH$.
 

\begin{definition}[Falsification Problem]
Given a PHIOA automaton $\AutomatonH$ and a safety specification $\varphi$, a falsification problem is to find the sets of input parameters values $P_0$ of $\P$ and initial conditions of state variables $X_0$ of $\X$ that generates a counterexample driving the automaton $\AutomatonH$ toward a violation of $\varphi$.
\end{definition}

As a counterexample returned by addressing the falsification problem illustrates unexpected behaviors of a system, it reveals how a system should be repaired to mitigate an adversarial attack or accommodate a system change. The designer then either develops a corresponding pattern or search through the space of potential edits to transform an original system to a new system that produces no counterexample with respect to the given safety specification. %Let $\varphi$,$\pi_P \in \exec(\AutomatonH)$ be a counterexample returned from solving the falsification problem of $\AutomatonH$ against $\varphi$ w.r.t the set of input parameters values $P$, the designer can specify a resilient pattern $\Gamma$ based on $\pi_P$,   

\begin{definition}[Model Transformation Problem]
Given an initial PHIOA automaton $\AutomatonH$ with a set of parameter $\P$, a safety specification $\varphi$ such that $\AutomatonH \not\models \varphi$, and a resilient pattern $\Gamma$, addressing the model transformation problem of $\AutomatonH$ is to determine a new PHIOA automaton $\tilde{\AutomatonH} \deq \Gamma(\AutomatonH)$ such that $\tilde{\AutomatonH}\models\varphi$.
\end{definition}


%\begin{definition}[Model Synthesis Problem]
%Given an transformed PHIOA automaton $\tilde{\AutomatonH}$ and a safety specification, let $\varphi$,$\pi_P \in \exec(\AutomatonH)$ be a counterexample returned from solving the falsification problem of $\AutomatonH$ against $\varphi$ w.r.t the set of input parameters values $P$, and a resilient pattern $\Gamma$, addressing the model synthesis problem of $\AutomatonH$ is to determine a new PHIOA automaton $\tilde{\AutomatonH} \deq \Gamma(\AutomatonH,\pi_P)$ such that $\tilde{\AutomatonH}\models\varphi$.
%\end{definition}

In this paper, the transformed automaton $\tilde{\AutomatonH}$ contains a set of parameter $\P'= \P \cup \P_s$ in which the values of $\P_s$ need to be synthesized to ensure the system is correct. 

%\begin{definition}[Model Synthesis Problem]
%Given a transformed PHIOA automaton $\tilde{\AutomatonH}$ with set of parameter $\P_s$ and a safety specification $\varphi$, the model synthesis problem of $\AutomatonH$ is to determine a set of instance values $P_0$ of $\P'$ such that $\tilde{\AutomatonH} \models\varphi$.
%\end{definition}


%\begin{definition}[Model Synthesis Problem]
%Given an transformed PHIOA automaton $\tilde{\AutomatonH}$ and a safety specification, let $\varphi$,$\pi_P \in \exec(\AutomatonH)$ be a counterexample returned from solving the falsification problem of $\AutomatonH$ against $\varphi$ w.r.t the set of input parameters values $P$, and a resilient pattern $\Gamma$, addressing the model synthesis problem of $\AutomatonH$ is to determine a new PHIOA automaton $\tilde{\AutomatonH} \deq \Gamma(\AutomatonH,\pi_P)$ such that $\tilde{\AutomatonH}\models\varphi$.
%\end{definition}

%Recently, falsification tools such as Breach~\cite{donze2010breach} and S-taliro~\cite{annpureddy2011s} have been introduced and successfully applied to falsify complex automotive control systems. 



\subsection{Continuous-time Stateflow Chart}
In this paper, we model a PHIOA as a \emph{continuous-time} Stateflow chart, which is a commercial modeling language for hybrid systems integrated within MathWorks Simulink.
%
Continuous-time Stateflow chart\footnote{In this paper, we focus only on continuous-time Stateflow diagram that does not include hierarchical states.} supplies methods for engineers to quickly model as well as efficiently refine, test, and generate code for hybrid systems. In a Stateflow diagram, a designer can specify different data types including continuous state variables, parameters, inputs, and outputs of the model, and define the discrete structures similar to a PHIOA.
%
While the syntactic components of a continuous-time Stateflow chart are described similar to a hybrid automaton, there are slight differences between their semantics as a Stateflow diagram is deterministic and has urgent transitions with priorities. Intuitively, transitions in a Stateflow model is triggered as soon as the transition guard condition is satisfied, while a hybrid automaton can stay at the current mode as long as its invariant still holds. To overcome this gap, a recent work proposed in~\cite{bak2017hybrid} provides an equivalent translation for both classes of deterministic and non-deterministic hybrid automata to Stateflow diagrams. Other significant research have been done to translate back and forth between hybrid automata and Simulink/Stateflow models~\cite{alur2008symbolic,manamcheri2011step,minopoli2016sl2sx}.
%

\subsection{Breach}
%
%
In our proposed methodologies, we incorporated Breach into the model synthesizer of \toolreaffirm as an analysis mechanism to search for the counterexamples of hybrid systems. Given a PHIOA modeled as a continuous-time Stateflow chart, an STL specification represented the safety property, and some parameter ranges, Breach~\cite{donze2010breach} can perform an optimized search over parameter domains to find parameter values that cause the system violating the given STL specification. On the other hand, Breach can compute the sensitivity of execution traces to the initial conditions, which can be used to obtain completeness results by performing systematic simulations. We note that although falsification cannot completely prove the system correctness, it can efficiently find bugs existing in the initial design of CPS that are too complex to be formally verified~\cite{kapinski2015simulation}. These bugs are essential for an engineer to specify resilient patterns to repair the model. 
%
Moreover, the general problem of verifying a CPS modeled as a hybrid system is proved to be \emph{undecidable}~\cite{henzinger1995s}. Instead, the falsification algorithms embedded within Breach are scalable and work properly for black-box hybrid systems with different classes of dynamics.
%
Thus, in practice, engineers prefer to use counterexamples obtained by a falsification tool to refine their design. Our prototype \toolreaffirm utilizes the advantages of Simulink/Stateflow modeling framework and the falsification tool Breach to design a resilient pattern and perform the model synthesis for a resilient CPS.
%Since Mathworks Simulink/Statflow is a well-established modeling framworke for CPSs,
%\subsection{Falsification}

%To perform a model synthesis for CPS, a general approach would be to formally specify safety properties of a CPS that protect the system against possible adversarial attacks using formalisms such as signal temporal logic (STL)~\cite{maler2004monitoring} and to then iteratively improve the design using \emph{falsification}~\cite{nghiem2010monte}, which would automatically identify vulnerabilities in the design. 

%Formally speaking, given a CPS $\Sigma$ modeled as hybrid automata or Simulink/Stateflow models and a set of properties $\varphi$, a falsification problem is to find sets of input parameters $P\in \P$ and initial conditions of state variables $S_0 \in \S$ that drive the model toward violations of $\varphi$, \ie $\phi(\Sigma, P, S_0)\not\models\varphi$, where $\phi(\Sigma, P, S_0)$ is the behavior of system $\Sigma$ under the set of input parameters $\hat{P}$ and initial conditions $S_0$~\cite{kapinski2015simulation}.





